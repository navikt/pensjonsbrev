%! Author = H162135
%! Date = 29.06.2021

% Preamble
%! suppress = DocumentclassNotInRoot
\documentclass[12pt]{pensjonsbrev_v2}
\usepackage{listings}
\usepackage{lipsum}
\begin{document}
    \begin{letter}{\brevparameter}
        \tittel{Vi har innvilget søknaden din om 100 prosent alderspensjon}
        \lettersectiontitle{Vedtak}
        \letterparagraph{Du får 20 513 kroner før skatt fra 1. juni 2021.}
        \letterparagraph{Alderspensjonen din utbetales innen den 20. hver måned. Du finner oversikt over utbetalingene dine på nav.no/utbetalinger}
        \letterparagraph{Du har også søkt om avtalefestet pensjon (AFP) i privat sektor, og du vil få et eget vedtak om dette.}
        \letterparagraph{Vedtaket er gjort etter folketrygdloven §§ 19-2 til 19-8, 19-10, 19-15, 20-2, 20-3, 20-9 til 20-14, 20-19 og 22-12.}

        \lettersectiontitle{Det er egne skatteregler for pensjon}
        \letterparagraph{Du bør endre skattekortet når du begynner å ta ut alderspensjon. Dette kan du gjøre selv på skatteetaten.no/pensjonist. Der får du også mer informasjon om skattekort for pensjonister. Vi får skattekortet elektronisk. Du skal derfor ikke sende det til oss.}
        \letterparagraph{På nav.no/dinpensjon kan du se hva du betaler i skatt. Her kan du også legge inn ekstra skattetrekk om du ønsker det. Dersom du endrer skattetrekket, vil dette gjelde fra måneden etter at vi har fått beskjed.}

        \lettersectiontitle{Alderspensjonen din reguleres årlig}
        \letterparagraph{Reguleringen skjer med virkning fra 1. mai og selve økningen blir vanligvis etterbetalt i juni. Du får informasjon om dette på utbetalingsmeldingen din. På nav.no kan du lese mer om hvordan pensjonene reguleres.}

        \lettersectiontitle{Du kan søke om å endre pensjonen din}
        \letterparagraph{Du kan ha mulighet til å ta ut 20, 40, 50, 60, 80 eller 100 prosent alderspensjon. Etter at du har begynt å ta ut alderspensjon, kan du gjøre endringer med 12 måneders mellomrom. Hvis du har høy nok opptjening, kan du ta ut100 prosent alderspensjon når du selv ønsker det. Du kan alltid stanse pensjonen.}
        \letterparagraph{Du kan bruke pensjonskalkulatoren på nav.no/dinpensjon for å se om du kan endre alderspensjonen din.}

        \lettersectiontitle{Arbeidsinntekt og alderspensjon}
        \letterparagraph{Du kan arbeide så mye du vil uten atalderspensjonen din blir redusert. Det kan føre til at pensjonen din øker.}
        \letterparagraph{Hvis du har 100 prosent alderspensjon, gjelder økningen fra 1. januar året etter at skatteoppgjøret ditt er ferdig.}
        \letterparagraph{Andre pensjonsordningerMange er tilknyttet en eller flere offentlige eller private pensjonsordninger som de har pensjonsrettigheter fra. Du bør kontakte de du har slike ordninger med for å undersøke hvilke rettigheter du kan ha. Du kan også undersøke med siste arbeidsgiver.}

        \lettersectiontitle{Du må melde fra om endringer}
        \letterparagraph{Hvis du får endringer i familiesituasjon, planlegger opphold i utlandet, eller ektefellen eller samboeren din får endringer i inntekten, kan det ha betydning for beløpet du får utbetalt fra NAV. I slike tilfeller må du derfor straks melde fra til oss. I vedlegget ser du hvilke endringer du må si fra om.}
        \letterparagraph{Hvis du har fått utbetalt for mye fordi du ikke har gitt oss beskjed, må du vanligvis betale tilbake pengene. Du er selv ansvarlig for å holde deg orientert om bevegelser på kontoen din, og du må melde fra om eventuelle feil til NAV.}

        \lettersectiontitle{Du har rett til å klage}
        \letterparagraph{Hvis du mener vedtaket er feil, kan du klage innen seks uker fra den datoen du mottok vedtaket. Klagen skal være skriftlig. Du finner skjema og informasjon på nav.no/klage.}
        \letterparagraph{I vedlegget får du vite mer om hvordan du går fram.}

        \lettersectiontitle{Du har rett til innsyn}
        \letterparagraph{Du har rett til å se dokumentene i saken din. I vedlegget får du vite hvordan du går fram.}
        \closing
    \end{letter}
\end{document}