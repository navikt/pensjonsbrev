%Document class definerer parametre for dokument klassen, og bør derfor fortsatt ligge i malen
\documentclass[version=last,
    foldmarks=false,
    addrfield=backgroundimage]{scrlttr2}
\usepackage[utf8]{inputenc}
\usepackage{graphicx}
\usepackage{listings}
\usepackage{xcolor}
\usepackage{times}
\usepackage[T1]{fontenc}
\newcommand{\feltfornavnmottaker}{Håkon}
\newcommand{\feltetternavnmottaker}{Testholmen}
\newcommand{\feltgatenavnmottaker}{Gladtestveien}
\newcommand{\felthusnummermottaker}{42}
\newcommand{\feltpostnummermottaker}{1337}
\newcommand{\feltpoststedmottaker}{Tosslo}
\newcommand{\feltfoedselsnummer}{13374212345}
\newcommand{\feltalderspensjonmaanedligbeloep}{Tosslo}
\newcommand{\feltoppgittprosent}{Tosslo}
\newcommand{\felttestfelt}{Tosslo}

\newcommand{\frasedinerettigheter}{\brevseksjon{Det er egne skatteregler for pensjon}
Du bør endre skattekortet når du begynner å ta ut alderspensjon. Dette kan du gjøre selv på
skatteetaten.no/pensjonist. Der får du også mer informasjon om skattekort for pensjonister. Vi får
skattekortet elektronisk. Du skal derfor ikke sende det til oss.}

\newcommand{\frasenyfrase}{\brevseksjon{Det er egne skatteregler for pensjon}
Du bør endre skattekortet når du begynner å ta ut alderspensjon. Dette kan du gjøre selv på
skatteetaten.no/pensjonist. Der får du også mer informasjon om skattekort for pensjonister. Vi får
skattekortet elektronisk. Du skal derfor ikke sende det til oss.}


\pagenumbering{arabic}
\setlength{\parindent}{0em}
\setlength{\parskip}{1em}
\setplength{locwidth}{200pt}
\setplength{toaddrheight}{170pt}
\setkomavar{date}{\today}
\setkomavar{fromname}[description ]{\small Returadresse: NAV Familie- og pensjonsytelser Steinkjer}
\setkomavar{fromaddress}[description ]{\small Postboks 6600 Etterstad, 0607 OSLO }
\setkomavar{toaddress}[description ]{
    \uppercase{\feltgatenavnmottaker \space \felthusnummermottaker
    \\ \feltpoststedmottaker, \feltpostnummermottaker }}
\setkomavar{addresseeimage}[description ]{ \includegraphics[width=2cm]{nav} }
\newcommand{\hentVedlegg}[1]{\par \input{vedlegg/#1}}
\newcommand{\tittel}[1]{\setkomavar{subject}{#1}\opening{}}
\newcommand{\infofield}[1]{{\color{gray}\footnotesize #1}}
\newcommand{\brevseksjon}[1]{\par \textbf{#1} \newline}
\newcommand{\brevparameter}{
    \feltfornavnmottaker \space \feltetternavnmottaker \newline
    \uppercase{\feltgatenavnmottaker \space \felthusnummermottaker}  \newline
    \feltpostnummermottaker \space \feltpoststedmottaker}

\setkomavar{location}{\noindent
    \infofield{NAVs saksnummer:} 25305007\\
    \infofield{Navn:} \feltfornavnmottaker \space \feltetternavnmottaker \\
    \infofield{Fødselsnummer:} 09015425897}



\begin{document}
    \begin{letter}{\brevparameter}
        \tittel{Vi har innvilget søknaden din om 100 prosent alderspensjon}
        \brevseksjon{Vedtak}
        Du får \feltalderspensjonmaanedligbeloep kroner før skatt fra 1. juni 2021.

        \par Alderspensjonen din utbetales innen den 20. hver måned. Du finner oversikt over utbetalingene
        dine på nav.no/utbetalinger

        \par Du har også søkt om avtalefestet pensjon (AFP) i privat sektor, og du vil få et eget vedtak om
        dette.

        \par Vedtaket er gjort etter folketrygdloven §§ 19-2 til 19-8, 19-10, 19-15, 20-2, 20-3, 20-9 til 20-14,
        20-19 og 22-12.

        \frasedinerettigheter

        \par På nav.no/dinpensjon kan du se hva du betaler i skatt. Her kan du også legge inn ekstra skattetrekk
        om du ønsker det. Dersom du endrer skattetrekket, vil dette gjelde fra måneden etter at vi har fått
        beskjed.

        \brevseksjon{Alderspensjonen din reguleres årlig}
        Reguleringen skjer med virkning fra 1. mai og selve økningen blir vanligvis etterbetalt i juni. Du
        får informasjon om dette på utbetalingsmeldingen din. På nav.no kan du lese mer om hvordan
        pensjonene reguleres.

        \brevseksjon{Du kan søke om å endre pensjonen din}
        Du kan ha mulighet til å ta ut 20, 40, 50, 60, 80 eller 100 prosent alderspensjon. Etter at du har
        begynt å ta ut alderspensjon, kan du gjøre endringer med 12 måneders mellomrom. Hvis du har
        høy nok opptjening, kan du ta ut 100 prosent alderspensjon når du selv ønsker det. Du kan alltid
        stanse pensjonen.

        \par Du kan bruke pensjonskalkulatoren på nav.no/dinpensjon for å se om du kan endre
        alderspensjonen din.

        \brevseksjon{Arbeidsinntekt og alderspensjon}
        Du kan arbeide så mye du vil uten at alderspensjonen din blir redusert. Det kan føre til at
        pensjonen din øker.

        Hvis du har 100 prosent alderspensjon, gjelder økningen fra 1. januar året etter at skatteoppgjøret
        ditt er ferdig.

        \brevseksjon{Andre pensjonsordninger}
        Mange er tilknyttet en eller flere offentlige eller private pensjonsordninger som de har
        pensjonsrettigheter fra. Du bør kontakte de du har slike ordninger med for å undersøke hvilke
        rettigheter du kan ha. Du kan også undersøke med siste arbeidsgiver.

        \brevseksjon{Du må melde fra om endringer}
        Hvis du får endringer i familiesituasjon, planlegger opphold i utlandet, eller ektefellen eller
        samboeren din får endringer i inntekten, kan det ha betydning for beløpet du får utbetalt fra NAV.
        I slike tilfeller må du derfor straks melde fra til oss. I vedlegget ser du hvilke endringer du må si
        fra om.

        \par Hvis du har fått utbetalt for mye fordi du ikke har gitt oss beskjed, må du vanligvis betale tilbake
        pengene. Du er selv ansvarlig for å holde deg orientert om bevegelser på kontoen din, og du må
        melde fra om eventuelle feil til NAV.

        \brevseksjon{Du har rett til å klage}
        Hvis du mener vedtaket er feil, kan du klage innen seks uker fra den datoen du mottok vedtaket.
        Klagen skal være skriftlig. Du finner skjema og informasjon på nav.no/klage.
        I vedlegget får du vite mer om hvordan du går fram.


        \brevseksjon{Du har rett til innsyn}

        \par Du har rett til å se dokumentene i saken din. I vedlegget får du vite hvordan du går fram.
        Har du spørsmål?

        \par Kontakt oss gjerne på nav.no eller på telefon 55 55 33 34.
        Hvis du oppgir fødselsnummeret ditt når du tar kontakt med NAV, kan vi lettere gi deg rask og god hjelp.

        \hentVedlegg{rettigheter}
        \hentVedlegg{plikter}
    \end{letter}
\end{document}