% used for title1 and title2

% minus parameter to length
% The stretch added by plus 10pt minus 10pt is very important for LaTeX to be able to adjust the position of the elements
% This makes latex to judge how bad it would be to break up content vs removing a little bit of space above a title
% If for example you are in the middle of a paragraph, and you are just missing 10 pt of space to fit it in the current page
% , subtracting a couple of pixels above each title instead of shipping the paragraph to the next page or breaking up the paragraph would be the best idea.

% plus
% plus, like minus allows LaTeX to add some space evenly above each title in the page, such that there isn't much room left on the bottom of the page.
% This both looks nicer, and mitigates having large empty spaces at the end of a page, which might make readers think the letter is over.

% space that will be added in case the previous element was not a title. Used for chapter like breaks.
\newlength{\spacebetweentitleandnontitle}
\setlength{\spacebetweentitleandnontitle}{\dimexpr(25pt-\parskip) plus 8pt minus 8pt}

% do not split the command with newlines. It will be included in the content.
\newcommand{\lettersectiontitle}[1]{\pagebreak[3]\ifdefstring{\previouselement}{title}{}{\addvspace{\spacebetweentitleandnontitle}}#1\nopagebreak\\[0pt]\addpenalty{10000}\addvspace{3pt}}

% Title 1
\newcommand{\lettersectiontitleone}[1]{\lettersectiontitle{\titleone #1}\setpreviouselement{title}}

% Title 2
\newcommand{\lettersectiontitletwo}[1]{\lettersectiontitle{\titletwo #1}\setpreviouselement{title}}